\documentclass{acmsiggraph}

\usepackage[scaled=.92]{helvet}
\usepackage{times}
\usepackage{parskip}
\usepackage{graphicx}
\usepackage{footmisc}
\usepackage{url}
\onlineid{0}

\title{Plotting the Single-Electron Solution to Schr\"{o}dinger's Equation with OpenCL}

\author{Tim Horton\thanks{e-mail: hortot2@rpi.edu}\\Rensselaer Polytechnic Institute}

\begin{document}

\maketitle

\begin{figure}
    \includegraphics[width=84.5mm]{320.png}
    \caption{The 3-2-0 orbital of the hydrogen atom}
\end{figure}

\section{Abstract}

The realm of physics provides many readily parallelizable algorithms --- often involving the simple evaluation of a function at an enormous number of points in space --- which also happen to produce attractive visualizations. One such visualization is that of the probability distribution of the single electron within a hydrogen atom (or He$^+$, Li$^{2+}$, and so on). By evaluating Schr\"{o}dinger's equation at millions of points, one can construct a representation of the likely location of the electron --- the orbital cloud, if you will. Luckily, the evaluation of this function at a particular point is entirely independent of nearby points, so it is effectively perfect for parallelization. We will use OpenCL to develop and benchmark (on a varied array of hardware, from dated CPUs to very modern GPUs) an implementation of this algorithm which will produce a simple --- but attractive --- image of the atomic orbital of the single-electron hydrogen atom.

\section{Introduction}

asdfasdfadsf asdf adsf asdf asdf adfasdf adsf adf asfd

asdfasdfadsf asdf adsf asdf asdf adfasdf adsf adf asfd

asdfasdfadsf asdf adsf asdf asdf adfasdf adsf adf asfd

asdfasdfadsf asdf adsf asdf asdf adfasdf adsf adf asfd

\section{Physics}

\subsection{Schr\"{o}dinger's Equation}

\[
\psi\left(n, l, m\right)=\psi_c\left(n, l\right)
\mathit{e}^{-r/na}
\left(\frac{2r}{na}\right)^l
\left[L_{n-l-1}^{2l+1}
    \left(\frac{2r}{na}\right)\right]
Y_l^m\left(\theta,\phi\right)
\]

\[
\psi_c\left(n, l\right)=\sqrt{\left(\frac{2}{na}\right)^3
    \frac{\left(n-l-1\right)!}{2n\left[\left(n+l\right)!\right]^3}}
\]

\subsection{Spherical Harmonics}

\[
Y_l^m\left(\theta,\phi\right)=\epsilon
\sqrt{\frac{\left(2l+1\right)}{4\pi}
    \frac{\left(l-\left|m\right|\right)!}{\left(l+\left|m\right|\right)!}}
\mathit{e}^{{\rm i}m\phi}
P^m_l\left(\cos\theta\right)
\]

\subsection{Laguerre and Legendre}

\section{Implementation}

\section{Hardware}

Benchmarks will be performed on a number of different computation devices across a few different computers:

\begin{itemize}

\item ATI Radeon 4890, 800$\times$850MHz, 1GB VRAM, 250\$\footnote{All hardware prices listed are approximate launch prices. Prices of older hardware, especially the Core 2 Quad, have dropped significantly since introduction.\label{fn:prices}}

\item Intel Core i7 620M, 2$\times$3333MHz, 4GB RAM, 332\$\footref{fn:prices}

\item Intel Core 2 Quad Q6600, 4$\times$3000MHz, 4GB RAM, 851\$\footref{fn:prices}

\item Intel Atom N270, 1$\times$1600MHz, 2GB RAM, 44\$\footref{fn:prices}

\end{itemize}

These machines run a variety of different operating systems (including Mac OS X, Windows, and Linux). Comparisons made later in this paper assume that each OS and the drivers and OpenCL implementation used within them are created equally; this is only somewhat reasonable, and should be kept in mind when interpreting results.

Also, it should be noted that while each core on a given CPU could potentially be evaluating samples in parallel, a GPU's cores are much more restricted (less general-purpose) and must work in small groups to accomplish their work. Therefore, given the kernel used for this project, the 4890 listed above only has 50 compute units (16 stream processors work together to compute a single sample).

\section{Results}

\subsection{Large- vs. Small-scale Parallelism}

\subsection{Strong Scaling}

\subsection{Cost Effectiveness of Hardware}

\section{Future Work}

\section{Conclusion}

\section{Code}

All of the code developed for this project is available under the two-clause BSD license, and is hosted on GitHub:

\url{http://github.com/hortont424/orbitals}

\url{git://github.com/hortont424/orbitals.git}

\bibliographystyle{acmsiggraph}
\nocite{*}
\bibliography{paper}

\end{document}